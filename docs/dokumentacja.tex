\documentclass[11pt,a4paper]{article}

\usepackage[T1]{fontenc}
\usepackage[polish]{babel}
\usepackage{lmodern}
\usepackage{microtype}
\usepackage{geometry}
\geometry{margin=2.5cm}

\usepackage{hyperref}
\hypersetup{
  colorlinks=true,
  linkcolor=blue,
  urlcolor=blue,
  citecolor=blue
}

\usepackage{enumitem}
\usepackage{listings}
\usepackage{xcolor}
\usepackage{amssymb}
\usepackage{graphicx}
\usepackage{float}

\lstdefinestyle{code}{
  basicstyle=\ttfamily\small,
  breaklines=true,
  columns=fullflexible,
  frame=single,
  rulecolor=\color{black!20},
  backgroundcolor=\color{black!2},
  keywordstyle=\color{blue!70!black},
  commentstyle=\color{green!40!black},
  stringstyle=\color{orange!60!black},
  literate={ą}{{\k a}}1 {ć}{{\'c}}1 {ę}{{\k e}}1 {ł}{{\l}}1 {ń}{{\'n}}1 {ó}{{\'o}}1 {ś}{{\'s}}1 {ż}{{\.z}}1 {ź}{{\'z}}1 {Ą}{{\k A}}1 {Ć}{{\'C}}1 {Ę}{{\k E}}1 {Ł}{{\L}}1 {Ń}{{\'N}}1 {Ó}{{\'O}}1 {Ś}{{\'S}}1 {Ż}{{\.Z}}1 {Ź}{{\'Z}}1
}

\title{Dokumentacja projektu\\Kalkulator punktów Riichi Mahjong}
\author{
  Dariusz Piwowarski \and
  Mateusz Łopaciński \and
  Hieronim Koc
}
\date{Styczeń 2026}

\begin{document}

\maketitle
\tableofcontents
\newpage

\section{Wprowadzenie}

Riichi Mahjong to japońska odmiana klasycznej gry Mahjong, w której gracze układają ręce z płytek (tiles) w celu osiągnięcia zwycięskich kombinacji. Podliczanie punktów po zakończeniu rozgrywki jest skomplikowane i wymaga znajomości wielu reguł, yaku (specjalnych kombinacji) oraz systemu punktacji opartego na Han i Fu.

Niniejsza aplikacja mobilna automatyzuje proces obliczania punktów poprzez wykorzystanie aparatu telefonu do rozpoznawania układu płytek na stole i automatycznego obliczania wyniku.

\section{Cel projektu}

Głównym celem projektu jest stworzenie mobilnej aplikacji wspomagającej podliczanie punktów w grze Riichi Mahjong poprzez:

\begin{itemize}[leftmargin=*]
  \item \textbf{Automatyzację procesu liczenia} --- eliminacja potrzeby ręcznego liczenia punktów,
  \item \textbf{Wykorzystanie rozpoznawania obrazu} --- analiza układu klocków na stole za pomocą kamery,
  \item \textbf{Dokładność obliczeń} --- zapewnienie poprawności wyników zgodnie z oficjalnymi regułami,
  \item \textbf{Ułatwienie gry} --- szybsze i wygodniejsze podliczanie wyników partii.
\end{itemize}

\section{Technologie i działanie aplikacji}

\subsection{Wykorzystane biblioteki}

\begin{itemize}[leftmargin=*]
  \item \textbf{React Native} --- wieloplatformowy framework do budowy aplikacji mobilnych (Android, iOS),
  \item \textbf{react-navigation} --- obsługa nawigacji pomiędzy ekranami (stack + dolne zakładki); definiuje przechodzenie między kalkulatorem, listą yaku i historią gier,
  \item \textbf{react-native-vision-camera} --- dostęp do kamery i strumienia wideo; dostarcza surowe klatki do modułu rozpoznawania płytek,
  \item \textbf{react-native-fast-tflite} --- integracja z modelem TensorFlow Lite; wykonuje inferencję YOLO na klatkach z kamery i zwraca detekcje płytek,
  \item \textbf{@shopify/react-native-skia} --- renderowanie graficzne na żywo; służy do rysowania ramek wokół wykrytych płytek na podglądzie z kamery,
  \item \textbf{riichi-ts} --- biblioteka odpowiedzialna za logikę liczenia punktów w Riichi Mahjong (wykrywanie yaku, liczenie Han/Fu, wynik końcowy),
  \item \textbf{@react-native-async-storage/async-storage} --- lokalne przechowywanie danych; dzięki niej zapisywana jest historia rozegranych partii,
  \item \textbf{fuse.js} --- wyszukiwanie z tolerancją błędów w liście yaku (np. filtrowanie po nazwie lub opisie),
  \item \textbf{react-native-reanimated} --- biblioteka do tworzenia płynnych animacji i interakcji (korzysta z Worklets),
  \item \textbf{react-native-unistyles} --- nowoczesny silnik stylów oparty na C++ zapewniający wysoką wydajność i obsługę motywów.
\end{itemize}

\subsection{Model uczenia maszynowego}

Aplikacja wykorzystuje model TensorFlow Lite (\texttt{model.tflite}) wytrenowany w podejściu YOLO do rozpoznawania płytek Mahjong.

\begin{itemize}[leftmargin=*]
  \item wejściem modelu są klatki z kamery przeskalowane do kwadratu (np. 640x640),
  \item wyjściem jest lista detekcji (klasa płytki, współrzędne, pewność),
  \item po stronie aplikacji wykonywany jest post-processing (m.in. Non-Maximum Suppression) oraz mapowanie detekcji na strukturę ręki używaną przez moduł liczący punkty.
\end{itemize}

\section{Działanie aplikacji krok po kroku}

Poniżej opisano typowy scenariusz użycia aplikacji od uruchomienia do zapisania wyniku.

\begin{enumerate}[leftmargin=*]
  \item \textbf{Ekran startowy / zakładka \emph{Calculate}} --- użytkownik widzi główny ekran kalkulatora z możliwością przejścia do skanera lub ręcznego budowania ręki.
  \item \textbf{Skanowanie stołu} --- użytkownik przechodzi do ekranu skanera, kieruje kamerę na stół; aplikacja w czasie rzeczywistym próbuje wykryć komplet ręki zwycięzcy oraz istotne elementy (meldy, Dora, itp.).
  \item \textbf{Potwierdzenie ręki} --- po udanym wykryciu użytkownik przechodzi do ekranu potwierdzenia, gdzie może poprawić rękę (np. zmienić pojedyncze płytki, ustawić wiatry, zaznaczyć Riichi).
  \item \textbf{Obliczenie punktów} --- po potwierdzeniu ręki aplikacja wywołuje moduł \texttt{riichi-ts}, który zwraca wynik (Han, Fu, punktacja, lista yaku).
  \item \textbf{Prezentacja wyniku} --- na ekranie wyniku użytkownik widzi czytelne podsumowanie: kto wygrał, jaką ręką, z jakich yaku składa się wynik oraz jak rozdzielono punkty między graczy.
  \item \textbf{Zapis do historii} --- wynik może zostać zapisany do lokalnej historii; dane trafiają do pamięci przy użyciu AsyncStorage.
  \item \textbf{Przeglądanie historii i yaku} --- z dolnych zakładek użytkownik może przejść do listy historii (przegląd poprzednich gier) lub do listy yaku (edukacyjny przegląd możliwych kombinacji wraz z filtrowaniem).
\end{enumerate}

\begin{figure}[h]
  \centering
  % \includegraphics[width=0.9\textwidth]{diagram-flow.png}
  \caption{Przykładowy diagram działania aplikacji.}
\end{figure}

\section{Funkcjonalności}

\subsection{Skanowanie i rozpoznawanie}

\begin{itemize}[leftmargin=*,label=--]
  \item \textbf{Skanowanie stołu kamerą} --- aplikacja wykorzystuje kamerę telefonu do analizy układu płytek,
  \item \textbf{Rozpoznawanie elementów ręki}:
  \begin{itemize}[leftmargin=*,label=\(\circ\)]
    \item układ płytek w ręce zwycięzcy (closed part),
    \item wyłożone meldy (pon, chi, kan) --- open part,
    \item płytka wygrywająca (ron/tsumo),
    \item wskaźniki Dora i Uradora,
    \item wiatry (round wind, seat wind),
    \item deklaracje Riichi,
    \item specjalne zdarzenia (Ippatsu, Under the Sea, Blessing of Heaven).
  \end{itemize}
\end{itemize}

\subsection{Kalkulacja punktów}

\begin{itemize}[leftmargin=*]
  \item \textbf{Automatyczne obliczanie} --- aplikacja oblicza punkty na podstawie rozpoznanej ręki,
  \item \textbf{Szczegółowy opis wyniku} --- wyświetlanie wszystkich czynników wpływających na wynik:
  \begin{itemize}[leftmargin=*]
    \item lista yaku (kombinacji specjalnych),
    \item liczba Han i Fu,
    \item końcowa wartość punktowa,
    \item informacja o tym, kto wygrał i od kogo punkty zostały pobrane.
  \end{itemize}
\end{itemize}

\subsection{Ręczna edycja}

\begin{itemize}[leftmargin=*]
  \item \textbf{Uzupełnianie danych} --- możliwość ręcznego uzupełnienia informacji, które nie zostały poprawnie rozpoznane,
  \item \textbf{Korekta rozpoznania} --- edycja wykrytych płytek i meldów,
  \item \textbf{Ustawienia dodatkowe} --- konfiguracja wiatrów, typu wygranej, deklaracji Riichi.
\end{itemize}

\subsection{Historia gier}

\begin{itemize}[leftmargin=*]
  \item \textbf{Zapisywanie wyników} --- automatyczne zapisywanie obliczonych wyników,
  \item \textbf{Przeglądanie historii} --- lista wszystkich zapisanych gier,
  \item \textbf{Szczegóły ręki} --- możliwość przejrzenia szczegółów każdej zapisanej ręki.
\end{itemize}

\subsection{Lista Yaku}

\begin{itemize}[leftmargin=*]
  \item \textbf{Katalog yaku} --- kompletna lista wszystkich dostępnych yaku w Riichi Mahjong,
  \item \textbf{Szczegóły yaku} --- opis każdego yaku, jego wartość w Han, warunki spełnienia,
  \item \textbf{Filtrowanie} --- możliwość filtrowania yaku według różnych kryteriów (rzadkość, kategoria).
\end{itemize}

\subsection{Ekran kalibracji (planowane)}

\begin{itemize}[leftmargin=*,label=--]
  \item \textbf{Dopasowanie do zestawu klocków} --- możliwość kalibracji systemu rozpoznawania do konkretnego zestawu płytek,
  \item \textbf{Uczenie wyglądu} --- aplikacja uczy się wyglądu poszczególnych płytek,
  \item \textbf{Rozpoznawanie znaczników} --- kalibracja sticków Riichi i znaczników wiatrów.
\end{itemize}

\section{Opis ekranów i widoków}

Ta sekcja służy do szczegółowego opisania kolejnych ekranów aplikacji oraz powiązanych z nimi funkcjonalności. Dla każdego ekranu przewidziano także miejsce na zrzut ekranu.

\subsection{Ekran główny kalkulatora (\emph{Calculate Home})}

\textbf{Opis:}
\begin{itemize}[leftmargin=*]
  \item pierwszy ekran widoczny po uruchomieniu aplikacji (w zakładce \emph{Calculate}),
  \item prezentuje skrót dostępnych opcji: skanowanie ręki kamerą lub ręczne budowanie ręki,
  \item może zawierać podsumowanie ostatnio policzonej ręki lub skróty do historii.
\end{itemize}

\textbf{Miejsce na zrzut ekranu:}
\begin{figure}[H]
  \centering
  % \includegraphics[width=0.7\textwidth]{calculate-home.png}
  \caption{Ekran główny kalkulatora (do uzupełnienia).}
\end{figure}

\subsection{Ekran skanera (\emph{Scanner Screen})}

\textbf{Opis:}
\begin{itemize}[leftmargin=*]
  \item wyświetla podgląd z kamery z nałożonymi ramkami na wykryte płytki,
  \item w tle działa model TensorFlow Lite wykonujący rozpoznawanie płytek i budujący wstępną reprezentację ręki,
  \item po wykryciu kompletnej ręki pozwala przejść dalej do etapu potwierdzania.
\end{itemize}

\textbf{Miejsce na zrzut ekranu:}
\begin{figure}[H]
  \centering
  % \includegraphics[width=0.7\textwidth]{scanner-screen.png}
  \caption{Ekran skanera z kamerą (do uzupełnienia).}
\end{figure}

\subsection{Ekran potwierdzenia ręki (\emph{Confirm Screen})}

\textbf{Opis:}
\begin{itemize}[leftmargin=*]
  \item prezentuje rękę wykrytą przez model: płytki na ręce, meldy, Dora itp.,
  \item umożliwia ręczne poprawki (zmiana pojedynczych płytek, dodanie/usunięcie meldów),
  \item pozwala ustawić opcje wpływające na punktację: wiatry stołu i gracza, typ wygranej (Ron/Tsumo), deklaracje Riichi, Ippatsu itp.
\end{itemize}

\textbf{Miejsce na zrzut ekranu:}
\begin{figure}[H]
  \centering
  % \includegraphics[width=0.7\textwidth]{confirm-screen.png}
  \caption{Ekran potwierdzenia wykrytej ręki (do uzupełnienia).}
\end{figure}

\subsection{Ekran kalkulatora / budowania ręki (\emph{Calculator Screen})}

\textbf{Opis:}
\begin{itemize}[leftmargin=*]
  \item alternatywa dla skanowania --- ręczne budowanie ręki za pomocą selektora płytek,
  \item użytkownik wybiera płytki do części zamkniętej i otwartej (meldy) oraz ustawia parametry rozdania,
  \item po zbudowaniu poprawnej ręki może przejść do obliczenia punktów.
\end{itemize}

\textbf{Miejsce na zrzut ekranu:}
\begin{figure}[H]
  \centering
  % \includegraphics[width=0.7\textwidth]{calculator-screen.png}
  \caption{Ekran ręcznego budowania ręki (do uzupełnienia).}
\end{figure}

\subsection{Ekran wyniku (\emph{Result Screen})}

\textbf{Opis:}
\begin{itemize}[leftmargin=*]
  \item wyświetla szczegółowy wynik obliczeń: Han, Fu, końcową liczbę punktów,
  \item prezentuje listę wykrytych yaku wraz z liczbą Han przypisaną każdemu z nich,
  \item pokazuje, jak punkty zostały rozdzielone pomiędzy graczy (kto od kogo płaci, różnica punktów),
  \item umożliwia zapisanie wyniku do historii.
\end{itemize}

\textbf{Miejsce na zrzut ekranu:}
\begin{figure}[H]
  \centering
  % \includegraphics[width=0.7\textwidth]{result-screen.png}
  \caption{Ekran prezentujący wynik obliczeń (do uzupełnienia).}
\end{figure}

\subsection{Ekrany historii (\emph{History List \& Detail})}

\textbf{Lista historii:}
\begin{itemize}[leftmargin=*]
  \item prezentuje listę zapisanych rozdań (np. z datą, liczbą punktów, skrótem yaku),
  \item umożliwia szybkie wyszukiwanie / filtrowanie (np. po dacie lub wyniku),
  \item po wybraniu pozycji otwierany jest ekran szczegółów.
\end{itemize}

\textbf{Szczegóły rozdania:}
\begin{itemize}[leftmargin=*]
  \item pokazują pełną rękę, listę yaku, parametry rozdania oraz dokładny rozkład punktów,
  \item mogą zawierać dodatkowe dane kontekstowe (np. kto był dealerem, który to był stół/runda).
\end{itemize}

\textbf{Miejsce na zrzuty ekranu:}
\begin{figure}[H]
  \centering
  % \includegraphics[width=0.7\textwidth]{history-list.png}
  \caption{Lista zapisanych rozdań (do uzupełnienia).}
\end{figure}

\begin{figure}[H]
  \centering
  % \includegraphics[width=0.7\textwidth]{history-detail.png}
  \caption{Szczegóły pojedynczego rozdania (do uzupełnienia).}
\end{figure}

\subsection{Ekrany listy yaku (\emph{Yaku List \& Detail})}

\textbf{Lista yaku:}
\begin{itemize}[leftmargin=*,label=--]
  \item zawiera wszystkie obsługiwane yaku wraz z krótkim opisem i liczbą Han,
  \item umożliwia filtrowanie i wyszukiwanie (np. wg kategorii, liczby Han, popularności).
\end{itemize}

\textbf{Szczegóły yaku:}
\begin{itemize}[leftmargin=*,label=--]
  \item opisują dokładne warunki spełnienia danego yaku,
  \item mogą zawierać przykładowe układy płytek ilustrujące dane yaku.
\end{itemize}

\textbf{Miejsce na zrzuty ekranu:}
\begin{figure}[H]
  \centering
  % \includegraphics[width=0.7\textwidth]{yaku-list.png}
  \caption{Lista yaku z możliwością filtrowania (do uzupełnienia).}
\end{figure}

\begin{figure}[H]
  \centering
  % \includegraphics[width=0.7\textwidth]{yaku-detail.png}
  \caption{Ekran szczegółów pojedynczego yaku (do uzupełnienia).}
\end{figure}

\section{Opis komponentów}

\subsection{Scanner}

Komponent odpowiedzialny za skanowanie i rozpoznawanie płytek. Wykorzystuje:
\begin{itemize}[leftmargin=*,label=--]
  \item \textbf{react-native-vision-camera} --- dostęp do kamery,
  \item \textbf{TensorFlow Lite} --- model rozpoznawania,
  \item \textbf{Skia} --- renderowanie detekcji na obrazie.
\end{itemize}

\textbf{Główne funkcje:}
\begin{itemize}[leftmargin=*,label=--]
  \item przetwarzanie klatek wideo w czasie rzeczywistym,
  \item wykrywanie płytek na obrazie,
  \item wizualizacja detekcji (prostokąty na wykrytych obiektach),
  \item automatyczne wykrywanie kompletnej ręki.
\end{itemize}

\subsection{HandBuilder}

Komponent do ręcznego budowania i edycji ręki. Pozwala na:
\begin{itemize}[leftmargin=*,label=--]
  \item dodawanie/usuwanie płytek,
  \item tworzenie meldów (pon, chi, kan),
  \item ustawianie opcji (wiatry, typ wygranej, Riichi),
  \item weryfikację poprawności ręki.
\end{itemize}

\subsection{Calculator}

Moduł obliczający punkty na podstawie ręki. Wykorzystuje bibliotekę \texttt{riichi-ts} do:
\begin{itemize}[leftmargin=*,label=--]
  \item walidacji ręki,
  \item identyfikacji yaku,
  \item obliczania Han i Fu,
  \item obliczania końcowej wartości punktowej.
\end{itemize}

\section{Rozpoznawanie obrazu}

\subsection{Model TensorFlow Lite}

Aplikacja wykorzystuje model YOLO w formacie TensorFlow Lite do rozpoznawania płytek Mahjong.

\textbf{Proces przetwarzania:}
\begin{enumerate}[leftmargin=*]
  \item \textbf{Przechwytywanie klatki} --- kamera przechwytuje obraz w rozdzielczości 1920x1080,
  \item \textbf{Przygotowanie danych} --- obraz jest przeskalowywany do 640x640 i konwertowany do formatu float32,
  \item \textbf{Inferencja} --- model przetwarza obraz i zwraca detekcje,
  \item \textbf{Post-processing} --- zastosowanie Non-Maximum Suppression do eliminacji duplikatów,
  \item \textbf{Interpretacja} --- konwersja detekcji na strukturę ręki (\texttt{Hand}).
\end{enumerate}

\subsection{Struktura detekcji}

Każda detekcja zawiera:
\begin{itemize}[leftmargin=*,label=--]
  \item pozycję (x, y),
  \item rozmiar (width, height),
  \item klasę (typ płytki),
  \item pewność (confidence score).
\end{itemize}

\subsection{Optymalizacja}

\begin{itemize}[leftmargin=*,label=--]
  \item \textbf{Worklets} --- przetwarzanie odbywa się w wątku natywnym, nie blokując UI,
  \item \textbf{Core ML} --- na iOS wykorzystywana jest delegacja Core ML dla lepszej wydajności,
  \item \textbf{Resize plugin} --- efektywne przeskalowywanie obrazów.
\end{itemize}

\section{Kalkulacja punktów}

\subsection{Biblioteka riichi-ts}

Aplikacja wykorzystuje bibliotekę \texttt{riichi-ts} do obliczania punktów. Biblioteka implementuje oficjalne reguły Riichi Mahjong.

\textbf{Wejście:}
\begin{itemize}[leftmargin=*]
  \item zamknięta część ręki (closed part),
  \item otwarte meldy (open part),
  \item wiatry (round wind, seat wind),
  \item płytka wygrywająca (tile taken),
  \item opcje specjalne (Riichi, Ippatsu, Double Riichi, itd.).
\end{itemize}

\textbf{Wyjście:}
\begin{itemize}[leftmargin=*]
  \item liczba Han,
  \item liczba Fu,
  \item końcowa wartość punktowa (ten),
  \item lista wykrytych yaku,
  \item informacja o błędach (jeśli ręka jest niepoprawna).
\end{itemize}

\subsection{Przykładowy wynik}

\begin{lstlisting}[style=code]
{
  error: false,
  fu: 30,
  han: 6,
  isAgari: true,
  ten: 12000,
  yaku: {
    chinitsu: 5,
    tanyao: 1
  },
  yakuman: 0
}
\end{lstlisting}

\section{Historia i przechowywanie danych}

\subsection{AsyncStorage}

Aplikacja wykorzystuje \texttt{@react-native-async-storage/async-storage} do lokalnego przechowywania danych.

\textbf{Przechowywane dane:}
\begin{itemize}[leftmargin=*]
  \item historia obliczonych rąk,
  \item szczegóły każdej ręki (płytki, meldy, opcje, wynik).
\end{itemize}

\subsection{Hook useHistory}

Custom hook \texttt{useHistory} zarządza operacjami na historii:
\begin{itemize}[leftmargin=*,label=--]
  \item zapisywanie nowych wyników,
  \item pobieranie listy zapisanych gier,
  \item pobieranie szczegółów konkretnej gry,
  \item usuwanie zapisanych gier.
\end{itemize}

\section{Testowanie}

\subsection{Framework testowy}

Projekt wykorzystuje \textbf{Jest} jako framework testowy.

\subsection{Uruchomienie testów}

\begin{lstlisting}[style=code]
yarn test
\end{lstlisting}

\subsection{Sprawdzanie jakości kodu}

\begin{lstlisting}[style=code]
# Linting
yarn lint

# Sprawdzanie typów
yarn typecheck

# Sprawdzanie zależności cyklicznych
yarn circular-dependency-check
\end{lstlisting}

\section{Podsumowanie}

Aplikacja \textbf{Riichi Mahjong Calculator} to kompleksowe rozwiązanie do automatyzacji obliczania punktów w grze Riichi Mahjong. Wykorzystuje zaawansowane technologie rozpoznawania obrazu (TensorFlow Lite) oraz sprawdzoną bibliotekę do obliczania punktów (\texttt{riichi-ts}), zapewniając graczom szybkie i dokładne podliczanie wyników.

\subsection{Główne osiągnięcia}

\begin{itemize}[leftmargin=*]
  \item Wieloplatformowa aplikacja mobilna (iOS, Android),
  \item Real-time rozpoznawanie płytek z kamery,
  \item Automatyczne obliczanie punktów zgodnie z oficjalnymi regułami,
  \item Intuicyjny interfejs użytkownika,
  \item Historia zapisanych gier,
  \item Kompletna lista yaku z opisami.
\end{itemize}

\subsection{Możliwe rozszerzenia}

\begin{itemize}[leftmargin=*]
  \item Ekran kalibracji dla różnych zestawów płytek,
  \item Eksport wyników do PDF,
  \item Tryb offline z pełną funkcjonalnością,
  \item Wsparcie dla różnych wariantów reguł,
  \item Statystyki i analityka gier.
\end{itemize}

\end{document}

